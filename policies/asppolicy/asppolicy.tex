\documentclass[11pt, includeaddress]{classes/cthit}
\usepackage{titlesec}
\usepackage{verbatimbox}

\titleformat{\paragraph}[hang]{\normalfont\normalsize\bfseries}{\theparagraph}{1em}{}
\titlespacing*{\paragraph}{0pt}{3.25ex plus 1ex minus 0.2ex}{0.7em}

\graphicspath{ {images/} }

\begin{document}

\title{Aspningspolicy}
\approved{2024--02--22}
\revisioned{2025--02--27}
\maketitle

\thispagestyle{empty}

\newpage

\makeheadfoot%

%Rubriksnivådjup
\setcounter{tocdepth}{2}
%Sidnumreringsstart
\setcounter{page}{1}
\tableofcontents

\newpage

\section{Syfte}
Denna policy är till för att ge riktlinjer och stöd till arrangörer av aspning gällande planering samt förväntat uppträdande kring aspningen. 
Detta gäller för samtliga sektionsorgan och arbetsgrupper.


\section{Att tänka på innan aspning}
\begin{itemize}
  \item Innan påbörjad aspperiod så ska sektionsorgan skriva ett syftningsdokument med lista på samtliga asparrangemang, en beskrivning av arrangemanget och vad de förväntas få ut av arrangemanget, en kravprofil samt potentiella intressekonflikter och skicka detta till styret och valberedningen.
  \item Schemat för aspperioden ska bestämmas gemensamt av samtliga sektionsorgan och föreningar som har aspning.
  \item Dagar som aspning ockuperar sektionslokal skall göras tillgängliga övriga arrangerande sektionsorgan senast 4 veckor innan början av ordinarie aspperiod.
\end{itemize}

\section{Behandling av aspar}
\begin{itemize}
  \item Person som arrangerar aspning eller medverkar på aspning av annan anledning än att aspa får ej missbruka sin maktposition på något sätt.
  \item Person som arrangerar aspning ska avstå från att inleda intim relation med asp.
  \item Aspar får inte favoriseras eller särbehandlas.
  \item Sektionsorgan bör inte föra diskussioner kring aspar och nomineringar med utomstående parter.
\end{itemize}

\section{Förhållningsregler för aspningstillfälle}

\begin{itemize}
  \item Det är inte tillåtet att lyfta fram sitt sektionsorgan på något annat sektionsorgans bekostnad, d.v.s. man får inte säga eller skriva något nedlåtande om något annat sektionsorgan.
  \item Aspningstillfällen skall marknadsföras till alla inom sektionen i samtliga årskurser, och
  alla skall uppmuntras lika mycket till att söka eller aspa.
  \item Ingen alkoholhets får förekomma under aspningsarrangemang. 
  \item Sektionsorgan måste arrangera minst ett nyktert aspningsarrangemang.
  \item Om alkohol erbjuds vid ett aspningsarrangemang så skall även motsvarande alkoholfria alternativ erbjudas.
  \item Sektionsaktiva har rätt att avvisa aspar som missköter sig från pågående asparrangemang, men endast styrelsen har rätt att stänga av aspar från aspningen i helhet. 
\end{itemize}

\section{Från övriga styrdokument}
\subsection{Reglemente}
Det åligger alla sektionsorgan att arrangera minst två arrangemang vars syfte är att
väcka intresse för och informera om organets verksamhet för nya sökande.

\section{Åtgärder}
Om någon skulle bryta mot aspningspolicyn så ska detta anmälas till styrelsen som då utreder och beslutar om eventuell påföljd.
Sektionens påföljder är förtecknade i uppförandepolicyn. 

\end{document}