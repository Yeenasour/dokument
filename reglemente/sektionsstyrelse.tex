\section{\STYRITFULL}
 
\subsection{Befogenheter}
\STYRIT{} handhar i överensstämmelse med sektionens stadga, reglemente och
styrdokument samt av sektionsmötet fattade beslut den verkställande
ledningen av sektionens verksamhet.
 
\subsection{Sammansättning}
\STYRIT{} består av
\begin{itemize}
	\item ordförande
	\item vice ordförande
	\item kassör
	\item sekreterare
	\item studerandearbetsmiljöombud (SAMO) 
	\item 0-3 ledamöter.
\end{itemize} 
 
\subsection{Rättigheter}
\STYRIT{} har full insyn i sektionens alla organ och äger rätt att deltaga i deras
möten med yttranderätt.
  
\subsection{Åligganden}

\subsubsection{Det åligger \STYRIT}

\begin{att}
	\item Verka för sammanhållningen mellan sektionsmedlemmarna och verka för deras gemensamma intressen.
	\item Leda sektionens arbete.
	\item Övervaka genomförandet av sektionsmötets beslut och se till att de verkställs.
	\item Lämna förslag, till sektionens valberedning, på representanter till nästkommande \STYRIT.
	\item Utse representanter till kårens utskott
	\item Ansvara för att sektionens kommunikationsplatform administreras.
	\item Ansvara för att sektionens mail och digitala verktyg administreras.
	\item Ansvara för att nya kommittémedlemmar läggs in i sektionens medlemssystem.
	\item Ansvara att förtroendeinvalda får utbildning om GDPR och ansvarsfull datahantering, samt se till att förtroendevalda får hjälp i frågor som rör GDPR.
\end{att}

\subsubsection{Det åligger \STYRIT{}s ordförande}
\begin{att}
	\item Leda \STYRIT{}s verksamhet.
	\item Föra sektionens talan då annat ej stadgats eller beslutats.
	\item Teckna sektionens firma.
\end{att}

\subsubsection{Det åligger \STYRIT{}s vice ordförande}
\begin{att}
	\item Överta ordförandens åligganden i ordförandes frånvaro.
	\item Kalla till möte med alla ordföranden i sektionens organ varje läsperiod.
\end{att}

\subsubsection{Det åligger \STYRIT{}s kassör}
\begin{att}
	\item Fortlöpande kontrollera sektionens räkenskaper och bokföring.
	\item Teckna sektionens firma.
	\item Upprätta budgetförslag till de i stadgan berörda sektionsmötena.
	\item Till varje sektionsmöte kunna redogöra för sektionens ekonomiska ställning.
	\item Informera nya revisorer och kassörer i kommittéer och studienämnd om sektionens bokförings- och redovisningssystem.
\end{att}

\subsubsection{Det åligger \STYRIT{}s SAMO}
\begin{att}
	\item Föra sektionens talan i frågor om psykosocial studie- och arbetsmiljö.
	\item Arbeta aktivt med att förbättra arbetsmiljön och företräda studenterna på sektionen i arbetsmiljöfrågor, bland annat genom Fysiska- och Psykosociala skyddsronder.
	\item Agera som sektionens studerande arbetsmiljöombud gentemot högskolan.
	\item Vägleda sektionsmedlemmar i vilka resurser som finns att tillgå för att upprätthålla en god studiesocial hälsa.
	\item Hänvisa vidare sektionsmedlemmar som söker stöd till lämplig resurs.
	\item Hantera ärenden med respekt för berörda parters integritet.
\end{att}
 
\subsubsection{Det åligger \STYRIT{}s sekreterare}
\begin{att}
	\item Föra protokoll vid styrelsemöte.
	\item Tillse att protokoll från såväl styrelsemöte som sektionsmöte anslås i enlighet med stadga.
	\item Tillse att material som inkommer till sektionen anslås eller på annat sätt förmedlas dem det berör.
	\item Handha \STYRIT{}s handlingar.
\end{att}

\subsection{Mandatperiod}
Invalda till \STYRIT{} betraktas som en del av \STYRIT{} och sektionen från tillträdet, 1:a juli, till avträdet, 1:a juli ett år efter tillträdet.

Det yttersta ansvaret för \STYRIT{}s verksamhet och ekonomi övergår till de nyinvalda den 1:a juli.
