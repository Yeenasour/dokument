\section{Valberedning}

\subsection{Inval}
Teknologsektionens valberedning väljs av sektionsmötet.

\subsection{Sammansättning}
Valberedningen består av ordförande samt, i reglemente, fastställt antal ledamöter.

\subsection{Uppdrag}
Valberedningens uppdrag är att sammanställa och presentera lämpliga kandidater
till sektionens styrelse, studienämnd och kommitteer.

\subsection{Begränsningar}
Medlemmar i valberedningen får ej söka förtroendepost på sektionen som valberedningen bereder.

\subsection{Anslag}
Valberedningens nomineringar skall korrekt anslås, enligt \paragraphref{sec:protokoll:anslagning}, minst tre läsdagar före sektionsmöte.

\subsection{Rättigheter}
Valberedningen äger rätt att i namn och emblem använda sektionens namn och dess symboler.

\subsection{Skyldigheter}
Valberedningen är skyldig att känna till och rätta sig efter sektionens stadga, reglemente, policies, samt övriga handlingar och beslut.